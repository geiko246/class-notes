
\section{Nonlinear Equations}

Our goal is to find real number(s) $x_*$ such that
\[f(x_*) = 0 \]
where $f(x)$ is a continuous function.

% \subsection{Example 1: Hitting a Target}

\begin{theorem}[Intermediate Value Theorem]
	If $f(x)$ is continuous on $[a,b]$ and $c$ is a constant such that
	\[f(a)\leq c \leq f(b) \quad or \quad f(b) \leq c \leq f(c)\]
	then there exists an $x^*\in[a,b]$ such that $f(x^*) =c$
\end{theorem}

\begin{lemma} 
	A solution to $f(x)=0$ exists if $a<x<b$ and $f(a)f(b)<0$.
	\begin{proof} 
		Either $f(a)$ or $f(b)$ is negative while the other must be positive by the assumed inequality. Then employ Intermediate Value Theorem.
	\end{proof}
\end{lemma}


By algorithm:
\begin{enumerate} 
    \item Choose midpoint $c$ between $a$ and $b$
    \item If $f(c)f(a) < 0$ then $c$ is the new $b$, otherwise $c$ is the new $a$
    \item Assess $|f(c) - 0| < \epsilon$ for accuracy
\end{enumerate}

\begin{example} 
    Consider $f(x)=\mathrm{sin}(x)-x+\frac{1}{3}$ , $0\leq x\leq \pi$. Show that there exists $x_*\in(0,\pi)$ s.t. $f(x_*) =0$.
    $f(0) = \frac{1}{3} > 0$
    $f(\pi) = -\pi + \frac{1}{3} >0$
    By IVT, $\exists x_* \in (0,\pi)$ s.t.$f(x_*)=0$.
\end{example}

% homework 1 posted today

\begin{example} 
   Show that there exists unique $x_*\in(0,1)$ s.t. $f(x_*)=0$ where $f(x)=e^{-x}-x$.
   
   By IVT, $x_*$ exists. To show uniqueness. there are several ways:
   \begin{enumerate} 
       \item Observe that $f'(x) < 0$ over $(0,1)$, so it is monotincally decreasing and thus 1-1.
       \item $f(x)=0 \leftrightarrow g(x)=e^{-x} =x$. This is known as a fixed point. 
   \end{enumerate}
\end{example}

\textbf{Claim} Some systematic numeraical methods for solving $f(x)=0$ are based on solving, instead, the equation $g(x)=x$.

\subsection{Introduction to fixed point methods}
Suppose $a\leq x\leq b$ and $(x)\in[a,b] \forall x \in [a,b]$. There exists $x_*\in[a,b]$ such that $g(x_*) =x_*$.

\begin{theorem} [Brower's Fixed Point Theorem]
   Suppose $g(x)$ is $C[a,b]$. Also assme that $[a,b]$ is mapped onto itself via $g$ i.e.e $g(x)]in[a,b] \forall x \in[a,b]$. Then, there exists $c\in[a,b]$ s.t. $g(c)=c$.
\end{theorem}

\begin{definition} 
    The point $c$ of theorem 1 is called ``fixed-point'' of function $g(x)$ 
\end{definition}

\begin{proof} 
    Define $f(x) = g(x) - x$. Assume without loss of generality, $a \leq g(a)$  and $b \geq g(b)$.

    At $x=a$, $f(a) -a \leq 0$

    At $x=b$, $f(b) -b \geq 0$

    $\Rightarrow f(a)f(b)\leq 0$. By IVT, $\exists c\in[a,b] \mathrm{s.t.} f(c)=0 \leftrightarrow g(c) = c$
\end{proof}

\begin{example} 
   Show that the eq. $f(x)=e^x =x^2 -3=0$ has a solution in $(1,2)$. Taking the log,
   \begin{align*} 
       x = \mathrm{ln}(x^2+3)
   \end{align*}
   Note that $g'(x) > 0 \quad \forall x\in [a,b]$
   Thus, $g$ is monotonically increasing. Show that $g(1)\leq g(x) \leq g(2)$
\end{example}

\textbf{Goal} Solve $f(x)=0$ by using (generating) sequence $\{x_n\}_{n=0}^\infty$ where $x_n \rightarrow x_*$ and $f(x_*)=0$.

\subsection{Review of Basic Methods}
\begin{enumerate} 
    \item Bisection Method $x_0 = \frac{a+b}{2}$
    \item Newton's Method. If $f(x)$ is continuously differentiable with $f'(x_n)=0 \forall n$ the scheme is defined by $x_{n+1} = x_n-\frac{f(x_n))}{f'(x_n)}$. Give $x_0\in \mathbb{R}$
    \item Secant Method \\ It follows from the scheme of Newtonn's method by replacing $f'(x_n)$ with a finite difference $c = \frac{f(x_n) - f(x_{n-1})}{x_n - x_{n-1}}$ \\ Scheme: $x_{n+1} = x_n - f(x_n)c$; $n=0,1$. Need two points $x_0,x_1$ as initial guess.
\end{enumerate}

\subsection{Rigorous Study of Iterative Methods}

\textbf{Idea} start with an initial guess $x_0$. Then, generate $\{x_n\}$, $x_n\rightarrow x_*$ as $n\rightarrow \infty$. For a fixed point, we can guess $x_{n+1}=g(x_n)$.

\textbf{Remark} If $x_n rightarrow c$ and $g$ is continuous then $\lim_{n\rightarrow \infty}x_{n+1} = \lim_{n\rightarrow \infty}g(x_n) = g(\lim_{n\rightarrow \infty}x_n)$


\begin{definition} 
    Suppose $g(x)$ is $C[a,b]$. Then $g(x)$ is called a contraction on $[a,b]$ if there is a real constant $L$ s.t.
    \begin{align*} 
        0<L<1 \quad \mathrm{and} \quad |g(x)-g(y)|< L |x-y| \forall x,y\in[a,b]
    \end{align*}
\end{definition}

\begin{remark} 
 By the contraction property, interval mapped by g shrinks $\epsilon_1 \leq L\epsilon < \epsilon$.
\end{remark}
% be able to type algorithms

%%%%%% Thursday, September 2

%%% Thursday, September 16

% Secant method theorem

\begin{definition} [Secant Method]
    The scheme:
    \begin{equation} 
        x_{n+1} = x_n - f(x_n)\frac{x_n - x_{n-1}}{f(x_n)-f(x_{n-1})}
    \end{equation}
\end{definition}

\begin{proof} 
    Suppose $f'(x) = \beta > 0$ (without loss of generality). Since $f'(x)$ is continuous in $I_h$ for any $\epsilon > 0$ we can choose $\delta>0$ s.t. for every $x$ in $I=[x_*-\delta,x_*+\delta]$ where $0\leq \delta\leq h$ we have
    \[|f'(x) - \beta| < \epsilon\]
    $\Leftrightarrow  \beta - \epsilon < f'(x) < \epsilon + \beta$

    Pick $\epsilon=\beta/4$: $3\beta/4 < f'(x) < 5\beta/4\, \forall x\in I_\delta$.

    Secant Method: 
    \[x_{n+1} = x_n - f(x_n)\frac{x_n - x_{n-1}}{f(x_n)-f(x_{n-1})}\]

    By the MVT: $f(x_n)-f(x_{n-1}) = f'(\xi)(x_n-x_{n-1})$ for some $\xi_n$ between $x_n$ and $x_0$.

    \begin{equation} 
        x_{n+1} - x_* = x_n - x_* - f'(\zeta)(x_n - x_*)\frac{1}{f'(\zeta)}
    \end{equation}

    \begin{equation} 
        x_{n+1} - x_* = x_n - x_* \left(1-\frac{f'(\zeta_n)}{f'(\xi_n)}\right)
    \end{equation}

    \begin{equation} 
        x_{n+1} - x_* \leq |x_n - x_*| \left(1-\frac{f'(\zeta_n)}{f'(\xi_n)}\right) = \frac{2}{3} |x_n-x_*|
    \end{equation}
\end{proof}

\subsubsection{Heuristic Derivation of $q=\frac{1+\sqrt{5}}{2}$}

Define the n-th interation error $e_n=x_n-x_*$. Assume $x_n \rightarrow x_*$. SM scheme: 

\[e_{n+1} = \frac{e_{n-1}f(x_n) - e_nf(x_{n-1})}{f(x_n)-f(x_{n-1})}\]

\begin{align}
    e_{n+1} = e_ne_{n-1}\frac{\frac{f(x_1)}{e_n}- \frac{f(x_{n-1})}{e_{n-1}}}{x_n - x_{n-1}} \frac{x_n-x_{n-1}}{f(x_n)-f(x_{n-1})} \label{eq1}
\end{align}

We will show $e_{n+1} \approx ce_ne_{n-1}$ for $n>>1$.

Taylor expansion:
% LEARN : 
% * learn how to do curly floating brackets above math
% * comment on line
\begin{align}
    f(x_n) = f(x_*) + (x_n -x_*) f'(x_*) + \frac{1}{2} (x_n-x_*)^2 f''(x_*) + \frac{1}{6} (x_n-x_*)^3f'''(\xi_n)\\
    \frac{f(x_n)}{e_n}= f'(x_*) + \frac{1}{2} (x_n-x_*) f''(x_*) + \frac{1}{6} (x_n-x_*)^2f'''(\xi_n) \\ 
    \frac{f(x_{n-1})}{e_{n-1}}= f'(x_*) + \frac{1}{2} (x_{n-1}-x_*) f''(x_*) + \frac{1}{6} (x_{n-1}-x_*)^2f'''(\xi_{n-1}) \\ 
    \frac{f(x_{n})}{e_{n}} - \frac{f(x_{n-1})}{e_{n-1}}= \frac{1}{2}(e_n - e_{n-1}) f''(x_*) + \frac{1}{6} \left[e_n^2f'''(\xi_{n}) - e_{n-1}^2f'''(\xi_{n-1})\right] \\ 
    % MEAN VALUE THEOREM
    e_{n+1}= e_ne_{n-1}\left\{ \frac{1}{2}f''(x_*) + \frac{1}{6} \frac{e_n^2f'''(\xi_{n}) - e_{n-1}^2f'''(\xi_{n-1})}{e_n - e_{n-1}}\right\} \frac{1}{f'(\sigma_n)}
\end{align}

Suppose 
\[\frac{1}{6} \frac{e_n^2f'''(\xi_{n}) - e_{n-1}^2f'''(\xi_{n-1})}{e_n - e_{n-1}} \rightarrow 0\]
Then for $n>1$ we can write $e_{n+1} \approx e_ne_{n-1}c$ where $c=\frac{1}{2}f''(x_*)$. Suppose that the limit $A \approx \frac{|e_{n+1}|}{|e_n|^q} $

\begin{align}
    A{|e_n|}^q \approx |e_{n+1}|\\
    |e_{n-1}| = A^{-1/q}|e_n|^{-1/q}
\end{align}

Combining with the bound,

\begin{align}
    A|e_{n+1}|^q \approx |e_n| A^{-1/q}|e_n|^{-1/q}\\
    |e_n|^{q-\frac{1}{q}-1}\approx |c|A^{-1-\frac{1}{q}}
\end{align}
 In order for the approximation to be satisfied (arguing that $e_n$ cannot go to infinity or zero if $q-\frac{1}{q}-1$ is greater or less than 0, called a scaling argument). So,

 \begin{align}
     q-\frac{1}{q}-1 = 0 \\
     q^2-q-1=0\\
     q= \frac{1+\sqrt 5}{2}
 \end{align}

 Must deal with:
\[\frac{1}{6} \frac{e_n^2f'''(\xi_{n}) - e_{n-1}^2f'''(\xi_{n-1})}{e_n - e_{n-1}} \rightarrow 0\]

Write numerator as

\begin{align}
    e_n^2f'''(\xi_{n}) - e_{n-1}^2f'''(\xi_{n-1}) = (e_n^2 - e_{n-1}^2)f'''(\xi_n) + e_n^2\left[f'''(\xi_n)- f'''(\xi_{n-1})\right] \\ 
\end{align}

So, 

\begin{align}
    \frac{1}{6} \frac{e_n^2f'''(\xi_{n}) - e_{n-1}^2f'''(\xi_{n-1})}{e_n - e_{n-1}} = \frac{1}{6}(e_n + e_{n-1})f'''(\xi_n)+e_n^2f^{(4)}(\bar \xi_n)\frac{\xi_n - \xi_{n-1}}{x_n - x_{n-1}}
\end{align}
A sufficient condition for the last term to approach 0 as $n\rightarrow \infty$ is

\[|\xi_n - \xi_{n-1}| \leq C|x_n - x_{n-1}|\]
Then
\[e_{n-1}^2 \leq |e_{n-1}|^2 \frac{C|x_n-x_{n-1}|}{e_n-e_{n-1}}\]

\section{Introduction to Polynomial Interpolation}

Consider $n+1$ pairs of points $(x_0,y_0),(x_1,y_1),\hdots,(x_n,y_n)$ where $x_i$ are distinct.

\textbf{Goal} is to construct a function $Q(x)$ from a known family of functions, e.g. polynomials, such that

\[Q(x_j)=y_j\quad \forall j\;0\leq j\leq n\]

\begin{definition} 
    The points $x_0,x_1,\hdots,x_n$ are ``interpolation points''. The requirement that $Q(x_j) = y_j$ where $y_j$ are ``data'', is also referred to as ``interpolating the data''.

    The function $Q(x)$ is the ``interpolant'' or the ``interpolating function''.
\end{definition}

\subsubsection{Notation}

In $Q_n(x)$, $n$ indicates the number of interpolation points (minus 1).

\textbf{Problem}: Given distinct points $x_0,\hdots,x_n$. \textbf{Find} a polynomial $Q_n(x)$ of the \textit{lowest} possible degree such that

\[Q(x_j)=y_j\]

\begin{theorem} 
If $x_0,x_1,\hdots,x_n \in \mathbb{R}$, then for any data points $y_0,\hdots,y_n \in \mathbb{R}$ there exists a unique polynomial $Q_n(x)$, $\mathrm{deg}(Q_n)\leq n$ such that $Q_n(x_j) = y_j$ for all $j$.
\end{theorem}

\begin{proof}[Uniqueness of $Q_n$]
    If $Q_n$ exists by construction, it has to be unique. Suppose $Q_n$ is not unique. Then there exists another polynomial $P_n(x),\; \mathrm{deg}(P_n) \leq n$ such that it meets every data point. Consider
    \[H_n(x) = P_n(x) - Q_n(x)\]
    Then $\mathrm{deg}(H_n)\leq n$; Also, $H_m(x_j)=0$ for $j=0,\hdots,n$ so it has $n+1$ zeros. However, the only polynomial whose degree is less than its number of zeros is the zero polynomial (corollary of the Fundamental Theorem of Algebra). So, $H_n(x) \equiv 0$.
\end{proof}

\begin{proof}[Existence of $Q_n$]
   By induction, start with $n=0$. Take $Q_0(x) = y_0 = Q(x)$.
   
   Given $Q_{n-1}(x)$, find a $Q_n(x)$ which satsifies $Q_n(x_n) = y_n$. Take

   \[Q_n(x) = Q_{n-1}(x) + c(x_n-x_0)\hdots(x_n-x_{n-1})\] 

   The idea being that other points $j=0,\hdots,n-1$ do not affect the new function. However, for point $x_n$,

   \[c = \frac{y_n - Q_{n-1}(x_n)}{(x_n-x_0)(x_n-x_1)\hdots(x_n-x_{n-1})}\]

   One can also write the entire process By

   \begin{equation} 
       Q_n(x) = a_0 + \sum^n_{i=1}a_i\left\{\prod_{k=0}^{i-1}(x_j-x_0) \right\}
   \end{equation}
\end{proof}

\textbf{Vandermonde Determinant}

Express $Q_n(x) = \sum^n_{k=0}b_kx^k$. Each $Q_n(x_j)=y_j$ is a constraint equation. In matrix form,

\begin{align} 
    \begin{pmatrix} 
        1 &x_0 &\hdots &x_0^n \\
        1 &x_1 &\hdots &x_1^n \\
        \vdots &\vdots &\ddots &\vdots \\
        1 &x_n &\hdots &x_n^n \\
    \end{pmatrix}
    \begin{pmatrix} 
        b_0 \\
        b_1 \\
        \vdots \\
        b_n
    \end{pmatrix} = 
    \begin{pmatrix} 
        y_0 \\ y_1 \\ \vdots \\ y_n
    \end{pmatrix}
\end{align}

This system has a unique solution if

\[\mathrm{det}\left(\begin{pmatrix} 
    1 &x_0 &\hdots &x_0^n \\
    1 &x_1 &\hdots &x_1^n \\
    \vdots &\vdots &\ddots &\vdots \\
    1 &x_n &\hdots &x_n^n \\
\end{pmatrix}\right) \neq 0\]

\begin{lemma} 
    \[\mathcal{D}_n = \prod_{i,j=0 | i>j}^{n}(x_i - x_j)\]
\end{lemma}