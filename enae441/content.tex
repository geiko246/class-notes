\let \vec \mathbf
\section{Definitions}
Details can be found in the appendix of the book.
\begin{itemize} 
	\item \textbf{Datetime} : 2021-09-02 09:45:23 [04:00] (the latter part is the timezone). Time zone is generally not used because we'll be using UTC.
	\item $\vec r_\bullet$ : Location at Epoch 
	\item J2000 Timekeeping system : Terrestrial time (approximately UTC)
	\item Julia date number of days from an epoch Jan 1 4713 BC @ noon
	\item MID Modified Julia Date JD - 2400000 epoch in 1858
	\item Angles
	\begin{itemize} 
		\item Radians : whenever an angle appears as a polynomial
		\item \textbf{Degrees} : Divides into \textbf{minutes} and \textbf{seconds}
		\item \textbf{Degree} (in decimal) : divides into \textbf{arcminutes} (1/60 degree) \textbf{arcseconds} (1/60 arcminute) 45 12' 30''
		\item Revolutions 
		\item Hours : 24 hours = 1 revolution
	\end{itemize}
	\item Three properties characterize a coordinate system
	\begin{itemize} 
		\item center
		\item orientation alignment of the axes
		\item type: cartesian or spherical polar
		\item principal axis is the first axis ($x$) 
		\item perifocal plane is plane of orbit
		\item Compass heading measured clockwise from North
		\item Earth coordinates measure radial distance form urface of the earth rather than the center
	\end{itemize}
\end{itemize}

% Thursday, September 16

\begin{definition} [Hill Clohssey Wiltshire Equations (HCW)]
	\[\ddot x -2x\dot y -3n^2x = f_x\]	
	\[\ddot y +2n \dot x = f_y\]	
	\[\ddot z - n^2z = f_z\]	
\end{definition}